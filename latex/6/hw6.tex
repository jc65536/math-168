% Uncomment for shell escape
% \RequirePackage{shellesc}

\documentclass{article}

% Font
\usepackage{mlmodern}

% Margins
\usepackage[margin=1in]{geometry}

% Math symbols, proof environments
\usepackage{amsmath, amsthm, amssymb}

% Use this package for matrices
\usepackage{array}

% Images and positioning
\usepackage{graphicx, float, tikz}

% Trees
\usepackage{forest}

% Plots
\usepackage{pgfplots}

\usepackage{xcolor}

\usepackage{parskip}
\usepackage[T1]{fontenc}

% My definitions
\usepackage{mathdefs}

\title{Math 168 Homework 6}

\author{Jason Cheng}

\date{\today}

\begin{document}

\maketitle

\subsection*{Exercise 1}

\begin{enumerate}
  \item[(a)]
  \begin{gather*}
    p_q = \frac{1}{2} (q p_{q - 1} - (q + 1) p_q) \\
    p_0 = 1 - \frac{p_0}{2}
  \end{gather*}

  \item[(b)]
  \begin{gather*}
    g_0'(z) = \sum_{q = 0}^{\infty} q p_q z^{q - 1} \\
    zg_0'(z) = \sum_{q = 0}^{\infty} q p_q z^q \\
    zg_0'(z) + g_0(z) = \sum_{q = 0}^{\infty} (q + 1) p_q z^q \\
    \begin{aligned}
      (z - 1)(zg_0'(z) + g_0(z)) &= \sum_{q = 0}^{\infty} (q + 1) p_q z^q (z - 1) \\
      &= \sum_{q = 0}^{\infty} ((q + 1) p_q z^{q + 1} - (q + 1) p_q z^q) \\
      &= \sum_{q = 1}^{\infty} q p_{q - 1} z^q - \sum_{q = 0}^{\infty} (q + 1) p_q z^q \\
      &= \sum_{q = 1}^{\infty} q p_{q - 1} z^q - \sum_{q = 1}^{\infty} (q + 1) p_q z^q - p_0 \\
      &= \sum_{q = 1}^{\infty} (q p_{q - 1} - (1 + q) p_q) z^q - p_0 \\
      &= \sum_{q = 1}^{\infty} 2 p_q z^q - p_0 \\
      &= \sum_{q = 0}^{\infty} 2 p_q z^q - 2p_0 - p_0 \\
      &= \sum_{q = 0}^{\infty} 2 p_q z^q - 3p_0 \\
      &= 2g_0(z) - 2 && p_0 = 2 / 3 \\
    \end{aligned} \\
    \boxed{g_0(z) = 1 + \frac{1}{2} (z - 1)(z g_0'(z) + g_0(z))}
  \end{gather*}

  \item[(c)]
  \begin{align*}
    \frac{dh}{dz} &= \left( \frac{d}{dz} z^3 g_0(z) \right) (1 - z)^{-2} + z^3 g_0(z) \frac{d}{dz} (1 - z)^{-2} \\
    &= (3z^2 g_0(z) + z^3 g_0'(z)) (1 - z)^{-2} + z^3 g_0(z) (-2) (1 - z)^{-3} (-1) \\
    &= (3z^2 g_0(z) + z^3 g_0'(z)) (1 - z)^{-2} + 2z^3 g_0(z) (1 - z)^{-3} \\
    &= \frac{(3z^2 g_0(z) + z^3 g_0'(z))(1 - z) + 2z^3 g_0(z)}{(1 - z)^3} \\
    &= \frac{3z^2 g_0(z) + z^3 g_0'(z) - 3z^3 g_0(z) - z^4 g_0'(z) + 2z^3 g_0(z)}{(1 - z)^3} \\
    &= \frac{3z^2 g_0(z) + z^3 g_0'(z) - z^3 g_0(z) - z^4 g_0'(z)}{(1 - z)^3} \\
    &= \frac{z^2 (3g_0(z) + z g_0'(z) - z g_0(z) - z^2 g_0'(z)}{(1 - z)^3)} \\
    &= \frac{z^2 (2g_0(z) + g_0(z) + z g_0'(z) - z g_0(z) - z^2 g_0'(z))}{(1 - z)^3} \\
    &= \frac{z^2 (2g_0(z) + (1 - z) g_0(z) + (1 - z) z g_0'(z))}{(1 - z)^3} \\
    &= \frac{z^2 (2g_0(z) + (1 - z) g_0(z) + (1 - z) z g_0'(z))}{(1 - z)^3} \\
    &= \frac{z^2 (2g_0(z) - 2(g_0(z) - 1))}{(1 - z)^3} \\
    &= \frac{2z^2}{(1 - z)^3} \\
  \end{align*}

  \item[(d)]
  TODO
\end{enumerate}

\newpage

\subsection*{Exercise 2}

\begin{enumerate}
  \item[(a)]
  \begin{align*}
    Q &= \sum_{t \in \set{1, 2}} (e_t - a_t^2) \\
    &= (e_1 - a_1^2) + (e_2 - a_2^2) \\
    &= \left( \frac{r - 1}{n - 1} - \left( \frac{2(r - 1) + 1}{2(n - 1)} \right)^2
    \right) + \left( \frac{n - r - 1}{n - 1} - \left( \frac{2(n - r - 1) + 1}{2(n
    - 1)} \right)^2 \right) \\
    &= \frac{n - 2}{n - 1} - \frac{(2r - 1)^2}{4(n - 1)^2} - \frac{(2(n - r) -
    1)^2}{4(n - 1)^2} \\
    &= \frac{4(n - 1)(n - 2)}{4(n - 1)^2} - \frac{(2r - 1)^2}{4(n - 1)^2} - \frac{(2(n - r) - 1)^2}{4(n - 1)^2} \\
    &= \frac{4n^2 - 12n + 8}{4(n - 1)^2} - \frac{4r^2 - 4r + 1}{4(n - 1)^2} - \frac{4(n - r)^2 - 4(n - r) + 1}{4(n - 1)^2} \\
    &= \frac{4n^2 - 12n + 8}{4(n - 1)^2} - \frac{4r^2 - 4r + 1}{4(n - 1)^2} - \frac{4n^2 - 8nr + 4r^2 - 4n + 4r + 1}{4(n - 1)^2} \\
    &= \frac{-8n + 6}{4(n - 1)^2} - \frac{8r^2}{4(n - 1)^2} - \frac{-8nr}{4(n - 1)^2} \\
    &= \frac{-8n + 6 - 8r^2 + 8nr}{4(n - 1)^2} \\
    &= \frac{3 - 4n + 4rn - 4r^2}{2(n - 1)^2}
  \end{align*}

  \item[(b)]
  \begin{gather*}
    \frac{dQ}{dr} = \frac{4n - 8r}{2(n - 1)^2}
  \end{gather*}
  The critical value of this is \( r = n / 2 \), thus it is optimal to divide
  down the middle.
\end{enumerate}

\newpage

\subsection*{Exercise 3}

\newpage

\subsection*{Exercise 4}

\begin{enumerate}
  \item[(a)]
  Since there are no self loops, each triangle gives 2 3-paths from a node to
  itself for each of the 3 vertices, so the total number of triangles is \( 8
  \cdot 6 / 6 = \boxed{8} \).

  \item[(b)]
  The total number of 2-paths is \( 2 \cdot (2 \cdot 12 + 4 \cdot 3) = 72 \).
  The global clustering coefficient is \( 8 \cdot 6 / 72 = \boxed{2 / 3} \).
\end{enumerate}

\newpage

\subsection*{Exercise 5}

\begin{enumerate}
  \item[(a)]
  \begin{gather*}
    g_0(z) = \sum_{k = 0}^{\infty} p_k z^k \\
    g_0'(z) = \sum_{k = 0}^{\infty} k p_k z^{k - 1} \\
    g_0'(1) = \sum_{k = 0}^{\infty} k p_k = \angles{k} \\
    g_0''(z) = \sum_{k = 0}^{\infty} k(k - 1) p_k z^{k - 2} \\
    g_0''(1) = \sum_{k = 0}^{\infty} k(k - 1) p_k = \angles{k(k - 1)}
  \end{gather*}

  \item[(b)]
  Condition:
  \begin{gather*}
    \angles{k^2} > 2 \angles{k} \\
    \sum_{k = 0}^{\infty} k^2 p_k > 2 \sum_{k = 0}^{\infty} k p_k \\
    \sum_{k = 0}^{\infty} (k^2 - k) p_k > \sum_{k = 0}^{\infty} k p_k \\
    \angles{k^2 - k} > \angles{k} \\
    g_0''(1) > g_0'(1)
  \end{gather*}
  Find \( g_0', g_0'' \):
  \begin{align*}
    g_0'(z) &= p(-1)(1 - qz)^{-2}(-q) \\
    &= pq(1 - qz)^{-2} \\
    &= \frac{pq}{(1 - qz)^2} \\
    g_0'(1) &= \frac{pq}{(1 - q)^2} \\
    g_0''(z) &= pq(-2)(1 - qz)^{-3}(-q) \\
    &= \frac{2pq^2}{(1 - qz)^3} \\
    g_0''(1) &= \frac{2pq^2}{(1 - q)^3}
  \end{align*}
  \begin{gather*}
    \frac{2pq^2}{(1 - q)^3} > \frac{pq}{(1 - q)^2} \\
    \frac{2p(1 - p)^2}{p^3} > \frac{p(1 - p)}{p^2} \\
    2(1 - p) > p \\
    \boxed{p < \frac{2}{3}}
  \end{gather*}
\end{enumerate}

\newpage

\subsection*{Exercise 6}

\begin{enumerate}
  \item[(a)]
  Let \( X \) be a random variable representing the current degree, \( n \) be
  the total number of nodes. \( q_k \) is the probability that a neighbor has \(
  k \) excess degrees.
  \begin{gather*}
    q_k = \frac{n (k + 1) p_{k + 1}}{n \angles{k}}
  \end{gather*}
  Here, \( n(k + 1) p_{k + 1} \) is the total number of degrees that belong to
  nodes with degree \( k + 1 \), and \( n \angles{k} \) is the total degree over
  all nodes. The fraction represents the probability that an edge from the
  current node is attached to a degree \( k + 1 \) node. Simplify the
  expression, and we get
  \begin{gather*}
    q_k = \frac{(k + 1) p_{k + 1}}{\angles{k}}
  \end{gather*}

  \item[(b)]
  \begin{gather*}
    \angles{k} = 1 \cdot p + 3(1 - p) = 3 - 2p \\
    \begin{aligned}
      \sum_{k = 0}^{\infty} k q_k &= \sum_{k = 0}^{\infty} k \frac{(k + 1) p_{k + 1}}{\angles{k}} \\
      &= \frac{1(1 + 1)p + 3(3 + 1)(1 - p)}{3 - 2p} \\
      &= \frac{2p + 12 - 12p}{3 - 2p} \\
      &= \boxed{\frac{12 - 10p}{3 - 2p}}
    \end{aligned}
  \end{gather*}
\end{enumerate}

\newpage

\subsection*{Exercise 7}

\begin{enumerate}
  \item[(a)]
  Every node initially has degree \( n - 1 \). Later, each node gains an average
  of \( np \) degrees as it gains edges to the other cluster. Thus, the average
  degree is \( \boxed{n - 1 + np} \).

  \item[(b)]
  Initially, there are \( 2 \binom{n}{2} = n(n - 1) \) edges. Later, each of the
  \( n^2 \) candidates is added with uniform probability \( p \), so the
  expected number of edges is \( \boxed{n(n - 1) + n^2 p} \).

  \item[(c)]
  \begin{gather*}
    \begin{bmatrix}
      0 & 1 & 1 & 1 & p & p & p & p \\
      1 & 0 & 1 & 1 & p & p & p & p \\
      1 & 1 & 0 & 1 & p & p & p & p \\
      1 & 1 & 1 & 0 & p & p & p & p \\
      p & p & p & p & 0 & 1 & 1 & 1 \\
      p & p & p & p & 1 & 0 & 1 & 1 \\
      p & p & p & p & 1 & 1 & 0 & 1 \\
      p & p & p & p & 1 & 1 & 1 & 0 \\
    \end{bmatrix}
  \end{gather*}
  \begin{align*}
    Q &= \frac{1}{2m} \sum_{i,j} \left( A_{ij} - \frac{k_i k_j}{2m} \right) \delta_{g_i g_j} \\
    &= \frac{1}{2n(n - 1) + 2n^2 p} \sum_{i,j} \left( A_{ij} - \frac{n - 1 + np}{2n(n - 1) + 2n^2 p} \right) \delta_{g_i g_j} \\
    &= \frac{1}{2n(n - 1) + 2n^2 p} \sum_{i,j} \left( A_{ij} - \frac{1}{2n} \right) \delta_{g_i g_j} \\
    &= \frac{1}{2n(n - 1) + 2n^2 p} \left( \sum_{i,j} A_{ij} \delta_{g_i g_j} - \frac{1}{2n} \sum_{i,j} \delta_{g_i g_j} \right) \\
    &= \frac{1}{2n(n - 1) + 2n^2 p} \left( n(n - 1) - \frac{n}{2} \right) \\
    &= \frac{1}{2(n - 1) + 2n p} \left( n - 1 - \frac{1}{2} \right) \\
    &= \boxed{\frac{n - 3/2}{2(n - 1) + 2n p}}
  \end{align*}
\end{enumerate}

\end{document}
