% Uncomment for shell escape
% \RequirePackage{shellesc}

\documentclass{article}

% Font
\usepackage{mlmodern}

% Margins
\usepackage[margin=1in]{geometry}

% Math symbols, proof environments
\usepackage{amsmath, amsthm, amssymb}

% Use this package for matrices
\usepackage{array}

% Images and positioning
\usepackage{graphicx, float, tikz}

% Trees
\usepackage{forest}

% Plots
\usepackage{pgfplots}

\usepackage{xcolor}

\usepackage{parskip}
\usepackage[T1]{fontenc}

% Math commands
\newcommand{\R}{\mathbb{R}} % Real numbers
\newcommand{\Z}{\mathbb{Z}} % Integers
\newcommand{\C}{\mathbb{C}} % Complex numbers
\newcommand{\Pow}{\mathcal{P}}
\newcommand{\set}[1]{\left\{ #1 \right\}}
\newcommand{\setc}[2]{\left\{ #1 \middle| #2 \right\}}
\newcommand{\abs}[1]{\left| #1 \right|}
\newcommand{\var}{\mathrm{VAR}}
\newcommand{\ut}[1]{\text{ #1}}
\newcommand{\cov}{\mathrm{COV}}
\newcommand{\intR}{\int_{-\infty}^{\infty}}
\newcommand{\red}[1]{\textcolor{red}{#1}}
\newcommand{\blu}[1]{\textcolor{blue}{#1}}
\newcommand{\grn}[1]{\textcolor{green}{#1}}
\newcommand{\contradiction}{\Rightarrow\!\Leftarrow}
\newcommand{\fun}{\mathrm{Fun}}

\DeclareMathOperator{\pMat}{Mat}
\newcommand{\Mat}[2]{\pMat_{#1 \times #2}}
\DeclareMathOperator{\spn}{span}
\DeclareMathOperator{\im}{im}
\DeclareMathOperator{\rank}{rank}
\DeclareMathOperator{\nul}{null}
\DeclareMathOperator{\tr}{tr}
\newcommand{\angles}[1]{\left \langle #1 \right \rangle}
\newcommand{\conj}[1]{\overline{#1}}

\title{Math 168 Homework 4}

\author{Jason Cheng}

\date{\today}

\begin{document}

\maketitle

\subsection*{Exercise 1}

\begin{enumerate}
  \item[(a)]
  Given an E-R graph, we can choose \( \binom{n}{3} \) sets of 3 nodes. The
  probability that each set of 3 nodes is a triangle is \( p^3 \). Thus, the
  expected number of triangles is
  \begin{align*}
    \binom{n}{3} p^3 = \frac{n(n - 1)(n - 2)p^3}{6}
  \end{align*}
  In the limit of large \( n \), we have
  \begin{align*}
    \lim_{n \to \infty} \frac{n(n - 1)(n - 2)p^3}{6}
    &= \lim_{n \to \infty} \frac{(n - 1 - 1)p \cdot (n - 1 + 1)p \cdot (n - 1)p}{6} \\
    &= \lim_{n \to \infty} \frac{((n - 1)p - p)((n - 1)p + p)(n - 1)p}{6} \\
    &= \lim_{n \to \infty} \frac{(c - p)(c + p)c}{6} \\
    &= \lim_{n \to \infty} \frac{c^3 - cp^2}{6} \\
    &= \lim_{n \to \infty} \frac{c^3 - 0}{6} && \lim_{n \to \infty} cp^2 = 0 \\
    &= \frac{c^3}{6}
  \end{align*}

  \item[(b)]
  We can choose a set of 3 nodes in \( \binom{n}{3} \) ways and pick a middle
  node in 3 ways. For each set, the probability that they are a connected triple
  is \( p^2 \). The expected number of connected triples is
  \begin{align*}
    3 \binom{n}{3} p^2 &= \frac{1}{2} n(n - 1)(n - 2)p^2 \\
    &= \frac{1}{2} (n - 1 - 1)(n - 1 + 1)(n - 1)p^2 \\
    &= \frac{1}{2} ((n - 1)^2 - 1)(n - 1)p^2 \\
    &= \frac{1}{2} ((n - 1)^3 p^2 - (n - 1) p^2) \\
    &= \frac{1}{2} ((n - 1)c^2 - cp)
  \end{align*}

  In the limit of large \( n \), this becomes \( (n - 1)c^2 / 2 \), since \( cp
  \) goes to 0.

  \item[(c)]
  \begin{align*}
    \frac{3 \cdot c^3 / 6}{(n - 1)c^2 / 2} = \frac{c^3}{(n - 1)c^2} = \boxed{\frac{c}{n
    - 1}}
  \end{align*}
\end{enumerate}

\newpage

\subsection*{Exercise 2}

\begin{enumerate}
  \item[(a)]
  Total number of edges
  \begin{align*}
    3 \binom{n}{3} p &= 3 \binom{n}{3} \frac{c}{\binom{n - 1}{2}} \\
    &= 3 \cdot \frac{n(n - 1)(n - 2)}{6} \cdot \frac{2c}{(n - 1)(n - 2)} \\
    &= nc
  \end{align*}

  Sum of all node degrees would be \( 2nc \), so the expected degree would be \(
  2nc / n = \boxed{2c} \).

  \item[(b)]
  Every time we add a triangle, we add 2 degrees to each vertex of the triangle.
  Thus all nodes have even degree.

  Every node is considered for \( \binom{n - 1}{2} \) potential triangles
  independently. Each time it is considered, it has a probability of \( p \) to
  add 2 degrees. For each consideration, let \( X \) be the number of added
  degrees.
  \begin{gather*}
    p_X(2) = p \\
    p_X(0) = 1 - p
  \end{gather*}

  Aggregating the \( \binom{n - 1}{2} \) events, let \( Y \) be the number of
  degrees of a node.
  \begin{align*}
    p_Y(2k) &= \binom{n}{k} p^k (1 - p)^{n - k} \\
    &= \frac{n!}{k! (n - k)!} \left( \frac{c}{\binom{n - 1}{2}} \right)^k \left(
    1 - \frac{c}{\binom{n - 1}{2}} \right)^{n - k} \\
    &= \frac{c^k}{k!} \cdot \frac{n!}{(n - k)!} \cdot \left( \frac{2}{(n - 1)(n
    - 2)} \right)^k \left( 1 - \frac{2c}{(n - 1)(n - 2)} \right)^{n - k}
  \end{align*}
\end{enumerate}

\newpage

\subsection*{Exercise 3}

\end{document}
