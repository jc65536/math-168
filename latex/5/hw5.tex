% Uncomment for shell escape
% \RequirePackage{shellesc}

\documentclass{article}

% Font
\usepackage{mlmodern}

% Margins
\usepackage[margin=1in]{geometry}

% Math symbols, proof environments
\usepackage{amsmath, amsthm, amssymb}

% Use this package for matrices
\usepackage{array}

% Images and positioning
\usepackage{graphicx, float, tikz}

% Trees
\usepackage{forest}

% Plots
\usepackage{pgfplots}

\usepackage{xcolor}

\usepackage{parskip}
\usepackage[T1]{fontenc}

% Math commands
\newcommand{\R}{\mathbb{R}} % Real numbers
\newcommand{\Z}{\mathbb{Z}} % Integers
\newcommand{\C}{\mathbb{C}} % Complex numbers
\newcommand{\Pow}{\mathcal{P}}
\newcommand{\set}[1]{\left\{ #1 \right\}}
\newcommand{\setc}[2]{\left\{ #1 \middle| #2 \right\}}
\newcommand{\abs}[1]{\left| #1 \right|}
\newcommand{\var}{\mathrm{VAR}}
\newcommand{\ut}[1]{\text{ #1}}
\newcommand{\cov}{\mathrm{cov}}
\newcommand{\intR}{\int_{-\infty}^{\infty}}
\newcommand{\red}[1]{\textcolor{red}{#1}}
\newcommand{\blu}[1]{\textcolor{blue}{#1}}
\newcommand{\grn}[1]{\textcolor{green}{#1}}
\newcommand{\contradiction}{\Rightarrow\!\Leftarrow}
\newcommand{\fun}{\mathrm{Fun}}

\DeclareMathOperator{\pMat}{Mat}
\newcommand{\Mat}[2]{\pMat_{#1 \times #2}}
\DeclareMathOperator{\spn}{span}
\DeclareMathOperator{\im}{im}
\DeclareMathOperator{\rank}{rank}
\DeclareMathOperator{\nul}{null}
\DeclareMathOperator{\tr}{tr}
\newcommand{\angles}[1]{\left \langle #1 \right \rangle}
\newcommand{\conj}[1]{\overline{#1}}
\DeclareMathOperator{\cpoly}{char}
\newcommand{\innerp}[2]{\left \langle #1, #2 \right \rangle}
\newcommand{\norm}[1]{\left| \left| #1 \right| \right|}

\title{Math 168 Homework 5}

\author{Jason Cheng}

\date{\today}

\begin{document}

\maketitle

\subsection*{Exercise 2}

\begin{enumerate}
  \item[(a)]
  \newcommand{\kin}{k^{\text{in}}}
  \newcommand{\kout}{k^{\text{out}}}
  \newcommand{\akin}{\angles{\kin}}
  \newcommand{\akout}{\angles{\kout}}
  \( \sum \kin = \sum \kout \), since every in degree corresponds to an out
  degree on another node. We also have \( \sum \kin = n \akin \) and \( \sum
  \kout = n \akout \). Thus \( \akin = \akout \). Furthermore, \( \sum \kin +
  \sum \kout = 2m \), since the total degree is twice the number of edges.
  Therefore, \( 2m = n \akin + n \akout = 2n \akin \).

  \item[(b)]
  Assuming \( \kout_i, \kin_j \) are nonzero.

  The probability that a stub from node \( i \) is connected to node \( j \) is
  \[ \frac{\kin_j}{2m - 1} \]

  The total probability that node \( i \) is connected to node \( j \) is then
  \[ \frac{\kout_i \kin_j}{2m - 1} \approx \frac{\kout_i \kin_j}{2m} \]

  \item[(c)]
  The probability that two nodes \( i, j \) are reciprocal is
  \begin{align*}
    \frac{\kin_i \kout_i}{2m} \cdot \frac{\kin_j \kout_j}{2m}
  \end{align*}

  The expected number of reciprocal pairs is
  \begin{align*}
    &\frac{1}{2} \left( \sum_{i = 1}^{n} \sum_{j = 1}^{n} \frac{\kin_i \kout_i}{2m} \cdot \frac{\kin_j \kout_j}{2m} - \sum_{i = 1}^{n} \frac{(\kin_i \kout_i)^2}{4m^2} \right) \\
    &= \frac{1}{2} \left( \sum_{i = 1}^{n} \frac{\kin_i \kout_i}{2m} \sum_{j = 1}^{n} \frac{\kin_j \kout_j}{2m} - \sum_{i = 1}^{n} \frac{(\kin_i \kout_i)^2}{4m^2} \right) \\
    &= \frac{1}{2} \left( \sum_{i = 1}^{n} \frac{\kin_i \kout_i}{2n \akin} \sum_{j = 1}^{n} \frac{\kin_j \kout_j}{2n \akin} - \sum_{i = 1}^{n} \frac{(\kin_i \kout_i)^2}{4m^2} \right) \\
  \end{align*}
  \begin{align*}
    \sum_{j = 1}^{n} \frac{\kin_j \kout_j}{2n \akin}
    &= \frac{\sum_j \kin_j \kout_j}{2 \sum_{p} \kin_p}
  \end{align*}

  TODO
\end{enumerate}

\newpage

\subsection*{Exercise 3}

\begin{enumerate}
  \item[(a)]
  \begin{align*}
    q_k = \frac{(k + 1) p_{k + 1}}{\angles{k}}
  \end{align*}
  \begin{align*}
    g_1(z) &= \sum_{k = 0}^{\infty} q_k z^k \\
    &= \sum_{k = 0}^{\infty} \frac{(k + 1) p_{k + 1}}{\angles{k}} z^k \\
    &= \frac{1}{\angles{k}} \sum_{k = 0}^{\infty} (k + 1) p_{k + 1} z^k \\
    &= \frac{1}{\angles{k}} \sum_{k = 0}^{\infty} p_{k + 1} \left( \frac{d}{dz} z^{k + 1} \right) \\
    &= \frac{1}{\angles{k}} \frac{d}{dz} \left( \sum_{k = 0}^{\infty} p_{k + 1} z^{k + 1} \right) \\
    &= \frac{1}{\angles{k}} \frac{d}{dz} (g_0(z) - p_0 z^0) \\
    &= \frac{1}{\angles{k}} g_0'(z) \\
    &= \left( \sum_{k = 0}^{\infty} k p_k \right)^{-1} g_0'(z) \\
    &= \left( \sum_{k = 0}^{\infty} k p_k 1^{k - 1} \right)^{-1} g_0'(z) \\
    &= \left( g_0'(1) \right)^{-1} g_0'(z) \\
    &= \frac{g_0'(z)}{g_0'(1)}
  \end{align*}

  \item[(b)]
  \begin{align*}
    g_0(z) &= \sum_{k = 0}^{\infty} p_k z^k \\
    &= \sum_{k = 0}^{\infty} e^{-c} \frac{c^k}{k!} z^k \\
    &= e^{-c} \sum_{k = 0}^{\infty} \frac{(cz)^k}{k!} \\
    &= e^{-c} e^{cz} \\
    &= \boxed{e^{c(z - 1)}}
  \end{align*}

  \begin{align*}
    g_1(z) &= \frac{g_0'(z)}{g_0'(1)} \\
    &= \frac{e^{c(z - 1)} c}{e^{c(1 - 1)} c} \\
    &= \frac{e^{c(z - 1)} c}{c} \\
    &= e^{c(z - 1)}
  \end{align*}

  This means the excess degree distribution is equal to the degree distribution.
  In other words, there is no friendship paradox, since your friends are
  expected to have the same amount of friends as yourself.

  \item[(c)]
  \begin{enumerate}
    \item[(i)]
    \begin{align*}
      p_k =
      \begin{cases}
        1, & k = k_0 \\
        0, & k \ne k_0
      \end{cases}
    \end{align*}

    This is the unit impulse, whose pgf is \( g_0(z) = z^{k_0} \). Using the
    formula from part a, we get
    \begin{align*}
      g_1(z) &= \frac{g_0'(z)}{g_0'(1)} \\
      &= \frac{k_0 z^{k_0 - 1}}{k_0 \cdot 1^{k_0 - 1}} \\
      &= z^{k_0 - 1}
    \end{align*}
    And this pgf corresponds to the probability distribution
    \begin{align*}
      q_k =
      \begin{cases}
        1, & k = k_0 - 1 \\
        0, & k \ne k_0 - 1
      \end{cases}
    \end{align*}

    which claims that the excess degree of every node is \( k_0 - 1 \). The
    formula from part a is correct, and this makes sense, because if every node
    has \( k_0 \) neighbors, one would expect one's neighbors to have \( k_0 - 1
    \) other neighbors.

    \item[(ii)]
    If \( k_0 \ge 3 \), then we have
    \begin{gather*}
      u = g_1(u) \\
      u = u^{k_0 - 1} \\
      u = 0
    \end{gather*}
    We ignore the trivial solution \( u = 1 \).
    \begin{align*}
      S &= 1 - g_0(u) \\
      &= 1 - 0^{k_0} \\
      &= 1
    \end{align*}
  \end{enumerate}
\end{enumerate}

\newpage

\subsection*{Exercise 4}

\begin{enumerate}
  \item[(a)]
  Given a random edge, the probability that it leads to node \( i \) is
  \[ p(i) = \frac{k_i}{2m - 1} \approx \frac{k_i}{2m} \]

  Thus, the expected value of the destination node of a random edge is
  \begin{align*}
    \angles{x}_{\text{edge}} &= \sum_i x_i p(i) \\
    &= \frac{1}{2m} \sum_i k_i x_i
  \end{align*}

  \item[(b)]
  Note that
  \[ 2m = \sum_i k_i \]

  Thus
  \begin{align*}
    \angles{x}_{\text{edge}} - \angles{x} &= \frac{1}{\sum_i k_i} \sum_i k_i x_i - \angles{x} \\
    &= \frac{1}{\sum_i k_i} \sum_i k_i x_i - \frac{1}{n} \sum_i x_i \\
    &= \frac{1}{1 / n \cdot \sum_i k_i} \cdot \frac{1}{n} \sum_i k_i x_i - \frac{1}{n} \sum_i x_i \\
    &= \frac{\angles{kx}}{\angles{k}} - \angles{x} \\
    &= \frac{\angles{kx} - \angles{x} \angles{k}}{\angles{k}} \\
    &= \frac{\cov(k, x)}{\angles{k}} \\
  \end{align*}
\end{enumerate}

\newpage

\subsection*{Exercise 5}

\begin{enumerate}
  \item[(a)]
  \begin{align*}
    g_0(z) &= \sum_{k = 0}^{\infty} (1 - a) a^k z^k \\
    &= (1 - a) \frac{1}{1 - az} \\
    &= \frac{1 - a}{1 - az} \\
    g_0'(z) &= (1 - a)(-1)(1 - az)^{-2} (-a) \\
    &= \frac{a(1 - a)}{(1 - az)^2} \\
    g_0'(1) &= \frac{a}{1 - a} \\
    g_1(z) &= \frac{g_0'(z)}{g_0'(1)} \\
    &= \frac{(1 - a)^2}{(1 - az)^2}
  \end{align*}
  According to eq. 12.30, \( u = g_1(u) \), so
  \begin{gather*}
    u = g_1(u) \\
    u = \frac{(1 - a)^2}{(1 - au)^2} \\
    u(1 - 2au + a^2 u^2) = (1 - a)^2 \\
    a^2 u^3 - 2au^2 + u - (1 - a)^2 = 0
  \end{gather*}

  \item[(b)]
  Simply multiply the two factors to get the original equation:
  \begin{gather*}
    (u - 1)(a^2 u^2 - a(2 - a)u + (1 - a)^2) = 0 \\
    a^2 u^3 - a(2 - a)u^2 + (1 - a)^2 u - a^2 u^2 + a(2 - a)u - (1 - a)^2 = 0 \\
    a^2 u^3 + (-2a + a^2) u^2 + (1 - 2a + a^2) u - a^2 u^2 + (2a - a^2) u - (1 - a)^2 = 0 \\
    a^2 u^3 - 2au^2 + u - (1 - a)^2 = 0
  \end{gather*}
  Thus the nontrivial solution satisfies \( a^2 u^2 - a(2 - a)u + (1 - a)^2 = 0
  \).

  \item[(c)]
  \begin{align*}
    u &= \frac{a(2 - a) - \sqrt{a^2 (2 - a)^2 - 4a^2 (1 - a)^2}}{2a^2} \\
    &= \frac{a(2 - a) - \sqrt{(a(2 - a) - 2a(1 - a))(a(2 - a) + 2a(1 - a))}}{2a^2} \\
    &= \frac{a(2 - a) - \sqrt{a^2 (4a - 3a^2)}}{2a^2} \\
    &= \frac{2 - a - \sqrt{4a - 3a^2}}{2a}
  \end{align*}
  \begin{align*}
    S &= 1 - g_0(u) \\
    &= 1 - \frac{1 - a}{1 - (2 - a - \sqrt{4a - 3a^2}) / 2} \\
    &= 1 - \frac{1 - a}{1 - (1 - a / 2 - \sqrt{a - 3a^2 / 4})} \\
    &= 1 - \frac{1 - a}{a / 2 + \sqrt{a - 3a^2 / 4}} \\
    &= 1 - \frac{(1 - a)(a / 2 - \sqrt{a - 3a^2 / 4})}{a^2 / 4 - a + 3a^2 / 4} \\
    &= 1 - \frac{(1 - a)(a / 2 - \sqrt{a - 3a^2 / 4})}{a^2 - a} \\
    &= 1 + \frac{a / 2 - \sqrt{a - 3a^2 / 4}}{a} \\
    &= 1 + \frac{1}{2} - \sqrt{\frac{1}{a} - \frac{3}{4}} \\
    &= \frac{3}{2} - \sqrt{a^{-1} - \frac{3}{4}}
  \end{align*}

  \item[(d)]
  If \( a \le 1 / 3 \), then
  \begin{gather*}
    a^{-1} \ge 3 \\
    \sqrt{a^{-1} - \frac{3}{4}} \ge \frac{3}{2} \\
    S = \frac{3}{2} - \sqrt{a^{-1} - \frac{3}{4}} \le 0
  \end{gather*}
  So the GC would not exist (since the fraction of nodes in the GC is not
  positive).
\end{enumerate}

\newpage

\subsection*{Exercise 6}

\begin{enumerate}
  \item[(a)]
  The diameter would be the shortest path from one end of the circle to the
  opposite end. Since we can jump across \( c / 2 \) neighbors, this distance is
  \( (n / 2) / (c / 2) = \boxed{n / c} \).

  \item[(b)]
\end{enumerate}

\end{document}
