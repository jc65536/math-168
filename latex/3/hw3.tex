% Uncomment for shell escape
% \RequirePackage{shellesc}

\documentclass{article}

% Font
\usepackage{mlmodern}

% Margins
\usepackage[margin=1in]{geometry}

% Math symbols, proof environments
\usepackage{amsmath, amsthm, amssymb}

% Use this package for matrices
\usepackage{array}

% Images and positioning
\usepackage{graphicx, float, tikz}

% Trees
\usepackage{forest}

% Plots
\usepackage{pgfplots}

\usepackage{xcolor}

\usepackage{parskip}
\usepackage[T1]{fontenc}

% Math commands
\newcommand{\R}{\mathbb{R}} % Real numbers
\newcommand{\Z}{\mathbb{Z}} % Integers
\newcommand{\C}{\mathbb{C}} % Complex numbers
\newcommand{\Pow}{\mathcal{P}}
\newcommand{\set}[1]{\left\{ #1 \right\}}
\newcommand{\setc}[2]{\left\{ #1 \middle| #2 \right\}}
\newcommand{\abs}[1]{\left| #1 \right|}
\newcommand{\var}{\mathrm{VAR}}
\newcommand{\ut}[1]{\text{ #1}}
\newcommand{\cov}{\mathrm{COV}}
\newcommand{\intR}{\int_{-\infty}^{\infty}}
\newcommand{\red}[1]{\textcolor{red}{#1}}
\newcommand{\blu}[1]{\textcolor{blue}{#1}}
\newcommand{\grn}[1]{\textcolor{green}{#1}}
\newcommand{\contradiction}{\Rightarrow\!\Leftarrow}
\newcommand{\fun}{\mathrm{Fun}}

\DeclareMathOperator{\pMat}{Mat}
\newcommand{\Mat}[2]{\pMat_{#1 \times #2}}
\DeclareMathOperator{\spn}{span}
\DeclareMathOperator{\im}{im}
\DeclareMathOperator{\rank}{rank}
\DeclareMathOperator{\nul}{null}
\DeclareMathOperator{\tr}{tr}
\newcommand{\angles}[1]{\left \langle #1 \right \rangle}
\newcommand{\conj}[1]{\overline{#1}}

\title{Math 168 Homework 3}

\author{Jason Cheng}

\date{\today}

\begin{document}

\maketitle

\subsection*{Exercise 1}

Post: @73.

\newpage

\subsection*{Exercise 2}

Commented on post @70.

\newpage

\subsection*{Exercise 3}

\begin{enumerate}
  \item[(a)]
  The degree centrality of node 1 is \( \boxed{0} \), and the degree centrality
  of every other node is \( \boxed{1} \).

  \item[(b)]
  The adjacency matrix looks like this:
  \[
    A =
    \begin{bmatrix}
      0 & 0 & 0 & 0 \\
      1 & 0 & 0 & 0 \\
      0 & 1 & 0 & 0 \\
      0 & 0 & 1 & 0
    \end{bmatrix}
  \]

  This is triangular, so we see that the only eigenvalue is 0.

  The eigenvector associated with \( \lambda = 0 \) is \(
    \begin{bmatrix}
      0 & \dotso & 0 & 1
    \end{bmatrix}^t
  \)

  Thus the eigenvalue centrality of every node except node \( n \) is \(
  \boxed{0} \), and the eigenvalue centrality of node \( n \) is \( \boxed{1}
  \).

  \item[(c)]
  \begin{gather*}
    (I - \alpha A)x = \vec{1} \\
    \begin{bmatrix}
      1 & 0 & 0 & 0 \\
      -\alpha & 1 & 0 & 0 \\
      0 & -\alpha & 1 & 0 \\
      0 & 0 & -\alpha & 1 \\
    \end{bmatrix}
    x = \vec{1} \\
    \begin{bmatrix}
      x_1 \\
      -\alpha x_1 + x_2 \\
      -\alpha x_2 + x_3 \\
      -\alpha x_4 + x_4
    \end{bmatrix}
    = \vec{1} \\
    \begin{bmatrix}
      x_1 \\ x_2 \\ x_3 \\ x_4
    \end{bmatrix}
    =
    \begin{bmatrix}
      1 \\ 1 + \alpha \\ 1 + \alpha + \alpha^2 \\ 1 + \alpha + \alpha^2 + \alpha^3
    \end{bmatrix}
  \end{gather*}

  (we must choose \( \alpha < 1 \) for convergence)

  The Katz centrality of node \( i \) is
  \[ \boxed{
    \frac{1 - \alpha^i}{1 - \alpha}
  } \]

\end{enumerate}

\newpage

\subsection*{Exercise 4}

\begin{align*}
  C &= \frac{n}{\sum_{j} d_{ij}} \\
  &= \frac{n}{2 \cdot (1 + 2 + \dotsb + ((n - 1) / 2 - 1)) + (n - 1) / 2} \\
  &= \frac{n}{((n - 1) / 2)((n - 1) / 2 - 1) + (n - 1) / 2} \\
  &= \frac{n}{((n - 1) / 2)((n - 1) / 2)} \\
  &= \boxed{\frac{4n}{(n - 1)^2}}
\end{align*}

\newpage

\subsection*{Exercise 5}

Central node
\begin{align*}
  x_1 &= \sum_{st} n^1_{st} \\
  &= \sum_{st}
  \begin{cases}
    1, & s \ne t \lor s = t = 1 \\
    0, & \text{otherwise}
  \end{cases} \\
  &= \boxed{n^2 - n + 1}
\end{align*}

Other nodes
\begin{align*}
  x_i &= \sum_{st} n^i_{st} \\
  &= \sum_{st}
  \begin{cases}
    1, & s = i \lor t = i \\
    0, & \text{otherwise}
  \end{cases} \\
  &= \boxed{2n - 1}
\end{align*}

\newpage

\subsection*{Exercise 6}

For each node \( j \) in the left half, the shortest path to node 2 is \( d_{2j}
= d_{1j} + 1 \) since it must pass through the edge \( (1, 2) \). Likewise, for
each node \( k \) in the right half, the shortest path to node 1 is \( d_{1k} =
d_{2k} + 1 \).

\begin{align*}
  C_1 &= \frac{n}{\sum_{j} d_{1j} + \sum_{k} d_{1k}} \\
  &= \frac{n}{\sum_{j} d_{1j} + \sum_{k} (d_{2k} + 1)} \\
  &= \frac{n}{\sum_{j} d_{1j} + \sum_{k} d_{2k} + n_2} \\
  C_2 &= \frac{n}{\sum_{j} d_{2j} + \sum_{k} d_{2k}} \\
  &= \frac{n}{\sum_{j} (d_{1j} + 1) + \sum_{k} d_{2k}} \\
  &= \frac{n}{\sum_{j} d_{1j} + n_1 + \sum_{k} d_{2k}} \\
\end{align*}

\begin{align*}
  \frac{1}{C_1} + \frac{n_1}{n}
  &= \frac{\sum_{j} d_{1j} + \sum_{k} d_{2k} + n_2}{n} + \frac{n_1}{n} \\
  &= \frac{\sum_{j} d_{1j} + \sum_{k} d_{2k} + n_1}{n} + \frac{n_2}{n} \\
  &= \frac{1}{C_2} + \frac{n_2}{n}
\end{align*}

\qed

\newpage

\subsection*{Exercise 7}

\begin{enumerate}
  \item[(a)]
  \( (A^3)_{ii} \) is the number of paths of length 3 from node \( i \) to
  itself. A path of length 3 would be a triangle starting and ending at node \(
  i \). However, we overcount the number of triangles by a factor of 6, because
  for the same triangle we could count 6 different paths: for each of the 3
  vertices, we can count 2 paths in different directions.

  \item[(b)]
  Number of 2-paths:
  \begin{align*}
    \sum_{ij} (A^2)_{ij} - \sum_{i} (A^2)_{ii}
  \end{align*}

  \item[(c)]
  \begin{align*}
    C &= \frac{\sum_{i} (A^3)_{ii}}{\sum_{ij} (A^2)_{ij} - \sum_{i} (A^2)_{ii}} \\
    &= \boxed{0.6666666666666666}
  \end{align*}
\end{enumerate}

\newpage

\subsection*{Exercise 8}

Legend:

\(
  \begin{array}{ll}
    \text{x-axis} & n \\
    \text{y-axis} & C \\
    \text{Blue square} & p = 0.05 \\
    \text{Red triangle} & p = 0.1 \\
    \text{Black circle} & p = 0.2
  \end{array}
\)

\begin{tikzpicture}[scale=1.5]
\begin{axis}
  \addplot [
    scatter,
    only marks,
    point meta=explicit symbolic,
    scatter/classes={
      0.05={mark=square*,blue},
      0.1={mark=triangle*,red},
      0.2={mark=o,draw=black}% <-- don't add comma
    },
  ] table [meta=p] {
n C p
5 0 0.05
10 0 0.05
15 0 0.05
20 0 0.05
25 0 0.05
30 0 0.05
35 0.04838709677419355 0.05
40 0 0.05
5 0 0.1
10 0 0.1
15 0 0.1
20 0.16216216216216217 0.1
25 0.14285714285714285 0.1
30 0.1 0.1
35 0.08275862068965517 0.1
40 0.06687898089171974 0.1
5 0 0.2
10 0.3333333333333333 0.2
15 0.3 0.2
20 0.24444444444444444 0.2
25 0.23684210526315788 0.2
30 0.18202247191011237 0.2
35 0.18358208955223881 0.2
40 0.18271119842829076 0.2
  };
\end{axis}
\end{tikzpicture}

\end{document}
